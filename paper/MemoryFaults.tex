\subsection{Memory Faults}

Even the best digital system, with high-quality components and design techniques, may not be infallible to faults. Despite the title of this subsection, when the entire digital system (or software) is considered, there are three terms for computing fault and they have different meanings: failure, fault and error \cite{Nelson1990}. 

\begin{itemize}
    \item \textit{Error}: An error is a manifestation of a fault in a system, in which the logical state of an element differs from its intended value. An error occurs for a particular system state and input when an incorrect next state and/or output results.
    \item \textit{Fault}: A fault is an anomalous physical condition. Causes include design errors, manufacturing problems, damage, fatigue, or other deterioration. Faults resulting from design errors and external factors are especially difficult to model and protect against because their occurrences and effects are hard to predict. A fault in a system does not necessarily result in an error;
    \item \textit{Failure}: A failure denotes an element's inability to perform its functions because of error in the element itself or its environment, which in turn are caused by various faults;
\end{itemize}