\section{Materials and Methods}

We first state our problem, then describe our data, showing all its characteristics. Then, we provide our hypothesis, define the setup, and perform the testing. Finally, we collect and perform data and regression analysis.

\subsection{Problem Statement}

As introduced in the first section of this paper, sorting is a fundamental concept and essential for solving other problems. The content of memory location can change unexpectedly, i.e., faults may happen at any time. Considering this, the main objective of this work is to design experiments to answer the following research question:

\begin{itemize}
    \item \textit{RQ1:} How are sorting algorithms affected by memory faults?
\end{itemize}

\subsection{Variables}

For this experimental study, we assume that the dependent and independent variables are as shown in Table \ref{table-variables} below:

\begin{table}[htbp]
    \caption{Dependent and independent variables.}
    \begin{center}
    \begin{tabular}{|c|c|}
    \hline
    \multicolumn{2}{|c|}{\textbf{Variables}}\\
    \hline
    \textbf{\textit{Dependent}} & \textbf{\textit{Independent}} \\
    \hline
    Largest subarray size & Probability of failure \\
    Percentage of largest subarray size related to the original array & Array size \\
    Unordered elements size & Sorting algorithm \\
    Percentage of unordered elements size related to the original array & \\
    \hline
    \end{tabular}
    \label{table-variables}
    \end{center}
\end{table}

\subsection{Hypothesis}

We define a set of hypothesis to test and draw some conclusions. They are:

\setcounter{hyp}{-1}
\begin{hyp}[Test hypothesis] \label{hyp:a}There is no difference in memory faults between tested algorithms.\end{hyp}
\begin{hyp} \label{hyp:b}There are differences in memory faults between tested algorithms. \end{hyp}
\begin{hyp} \label{hyp:c} An algorithm is better than others considering all dependent variables. \end{hyp}

\begin{hyp} \label{hyp:d} ... \end{hyp}

\subsection{Experimental Setup}

\subsection{Development}

To conduct the proposed study we get a set of files with the basic setup and instructions, that was composed by:
\begin{itemize}
    \item A file that contains a sequence of integers that are the input data;
    \item Four files, one for each os those algorithms: quicksort, bubblesort, insertion sort, and mergesort, that are used to sorting the input data;
    \item An output file with the sorted data.
\end{itemize}

\subsection{Data Analysis}

\subsection{Conclusions}

From that, we develop a Python script to generate custom inputs. The sequence has two characteristics: (\textit{i}) the sequence size, and (\textit{ii}) the probability of failure. For this work, the generated input files are shown in the Table \ref{table-input-data} below.

\begin{table}[htbp]
    \caption{Generated input data.}
    \begin{center}
    \begin{tabular}{|c|c|c|c|}
    \hline
    \multirow{1}{*}{\textbf{Input}}&\textbf{Sequence Size}&\textbf{Probability of Failure} \\
    \hline
    Input A & 100 & 1\% \\
    Input B & 100 & 2\% \\
    Input C & 100 & 5\% \\
    \hline
    Input D & 1000 & 1\% \\
    Input E & 1000 & 2\% \\
    Input F & 1000 & 5\% \\
    \hline
    Input G & 10000 & 1\% \\
    Input H & 10000 & 2\% \\
    Input I & 10000 & 5\% \\
    \hline
    \end{tabular}
    \label{table-input-data}
    \end{center}
\end{table}

The input file looks like follow:

\begin{verbbox}[\mbox{}]
0.01 100 9 48 37 6 26 7 24 44 17 50 48 30 49 33 22 13 42 29 39 13 19 13 9 28 
34 1 33 27 14 45 48 40 11 17 6 50 9 44 20 16 37 45 23 14 38 29 10 49 44 46 35
45 15 2 22 1 46 40 8 48 23 23 32 35 3 15 8 36 17 24 27 48 28 5 28 50 44 4 25 
6 9 1 11 44 26 50 44 12 7 20 30 20 37 20 6 8 13 15 20 49
\end{verbbox}

\begin{figure}[hbtp]
    \centering
    \fbox{
    \theverbbox
    }
    \caption{Example of input file.}
    \label{input-file-example}
\end{figure}

Figure \ref{input-file-example} shows an example of an input file generated by the Python script. The first number of the sequence (\texttt{0.01}) is the probability of memory failure when sorting. The second number (\texttt{100}) means the size of the integers sequence used by sorting. The rest of the numbers indicates the sequence itself.