\section{Materials and Methods}

We first describe our data, showing all its characteristics. Then, we provide our hypothesis, define the setup, and perform the testing. Finally, we collect and perform data and regression analysis.

\subsection{Experimental Data}

To conduct the proposed study we get a set of files with the basic setup and instructions, that was composed by:
\begin{itemize}
    \item A file that contains a sequence of integers that are the input data;
    \item Four files, one for each os those algorithms: quicksort, bubblesort, insertion sort, and mergesort, that are used to sorting the input data;
    \item An output file with the sorted data.
\end{itemize}

Falar das variaveis dependentes e independentes...

From that, we develop a Python script to generate custom inputs. The sequence has two characteristics: (\textit{i}) the sequence size, and (\textit{ii}) the probability of failure. For this work, the generated input files are shown in the table \ref{table2} below.

\begin{table}[htbp]
    \caption{Generated input data.}
    \begin{center}
    \begin{tabular}{|c|c|c|c|}
    \hline
    \multirow{1}{*}{\textbf{Input}}&\textbf{Sequence Size}&\textbf{Probability of Failure} \\
    \hline
    Input A & 100 & 1\% \\
    Input B & 100 & 2\% \\
    Input C & 100 & 5\% \\
    \hline
    Input D & 1000 & 1\% \\
    Input E & 1000 & 2\% \\
    Input F & 1000 & 5\% \\
    \hline
    Input G & 10000 & 1\% \\
    Input H & 10000 & 2\% \\
    Input I & 10000 & 5\% \\
    \hline
    \end{tabular}
    \label{table2}
    \end{center}
\end{table}

The input file looks like follow:

\begin{verbbox}[\mbox{}]
0.01 100 9 48 37 6 26 7 24 44 17 50 48 30 49 33 22 13 42 29 39 13 19 13 9 28 
34 1 33 27 14 45 48 40 11 17 6 50 9 44 20 16 37 45 23 14 38 29 10 49 44 46 35
45 15 2 22 1 46 40 8 48 23 23 32 35 3 15 8 36 17 24 27 48 28 5 28 50 44 4 25 
6 9 1 11 44 26 50 44 12 7 20 30 20 37 20 6 8 13 15 20 49
\end{verbbox}

\begin{figure}[hbtp]
    \centering
    \fbox{
    \theverbbox
    }
    \caption{Example of input file.}
    \label{input-file-example}
\end{figure}

Figure \ref{input-file-example} shows an example of an input file generated by the Python script. The first number of the sequence (\texttt{0.01}) is the probability of memory failure when sorting. The second number (\texttt{100}) means the size of the integers sequence used by sorting. The rest of the numbers indicates the sequence itself.

\subsection{Hypothesis}

\subsection{Linear Regression}