\section{Introduction}
Technology is deeply introduced in people's quotidian supporting a massive number of tasks, for example: searching for a shared car, surfing on the web, sending a message to someone, automating the company's production or using the company's software. Nevertheless, most people don't know that devices are continually dealing with memory failures, faults and errors. These devices were made with large and inexpensive memories, which are also error-prone \cite{Finocchi2004}.

Software behavior may be affected by the problems mentioned before, especially those from memory. We have a memory fault when the correct value that should be stored in a memory location gets altered because of a soft failure. In particular, the content of a location can change unexpectedly, i.e., faults may happen at any time: real memory faults are indeed highly dynamic and unpredictable \cite{Hamdioui2003}.

In the beginning steps of software development, the designer has a general idea of the structure and functions. For each one of these, some algorithms will be produced or used. In the following stages, the outcome software (and its algorithms) will be tested and, then, delivered to the user. Different kinds of algorithms could be written or used in the software, and one of these is the sorting algorithms.

A good algorithm is that which gives satisfactory results for every range of data set. Sorting is a fundamental concept and important for solving other problems like is prerequisite for Binary Search. Sorting is often used in a large variety of critical applications and is a fundamental task that is used by most computers \cite{NitinArora}.

In this paper, we present a discussion about how these sorting algorithms, particularly Quicksort, Mergesort, Insertion Sort and Bubblesort, are affected by memory faults.